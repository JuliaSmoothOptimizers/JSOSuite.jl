
% JuliaCon proceedings template
\documentclass{juliacon}
\setcounter{page}{1}

\usepackage{amsmath}

\begin{document}

% **************GENERATED FILE, DO NOT EDIT**************

\title{My JuliaCon proceeding}

\author[1]{1st author}
\author[1, 2]{2nd author}
\author[2]{3rd author}
\affil[1]{University}
\affil[2]{National Lab}

\keywords{Julia, Optimization, Game theory, Compiler}

\hypersetup{
pdftitle = {My JuliaCon proceeding},
pdfsubject = {JuliaCon 2022 Proceedings},
pdfauthor = {1st author, 2nd author, 3rd author},
pdfkeywords = {Julia, Optimization, Game theory, Compiler},
}



\maketitle

\begin{abstract}

\texttt{JSOSuite.jl} is a new Julia package offering a user-friendly interface for continuous nonlinear optimization.
The solvers available cover unconstrained to generally-constrained, and least-squares problems.
This new package caters to practitioners as it does not require an understanding of the inner mechanism of solvers, but instead performs a cursory analysis of the problem to match it with an appropriate solver.

\end{abstract}

\section{Introduction}

\texttt{JSOSuite.jl} is a Julia \cite{bezanson2017julia} package to find a local solution of continuous nonlinear optimization problems of the form
\begin{equation}\label{eq:nlp}
    \underset{x \in \mathbb{R}^{n}}{\text{minimize}} \ f(x) \ \text{subject to} \ c_L \leq c(x)  \leq c_U, \ \ell \leq x \leq u,
\end{equation}
where \(f:\mathbb{R}^n\to\mathbb{R}\) and  \(c:\mathbb{R}^n\to\mathbb{R}^m\) are continuously differentiable, with \(c_L \in \left(\mathbb{R} \cup \{-\infty\} \right)^{m}\),  \(c_U \in \left(\mathbb{R} \cup \{+\infty\} \right)^{m}\),  \(\ell \in \left(\mathbb{R} \cup \{-\infty\} \right)^n\), and  \(u \in \left(\mathbb{R} \cup \{+\infty\} \right)^n\).
Bounds on the variables appear separately from other types of constraints because numerical methods often treat them differently.

\section{Statement of Need}

\texttt{JSOSuite.jl} is part of the JuliaSmoothOptimizers (JSO) ecosystem, an academic organization that offers a collection of Julia packages for nonlinear optimization software development, testing, and benchmarking.

JSO provides a general API for solvers to interact with models by providing flexible data types to represent the objective and constraint functions, evaluate their derivatives, and provide essentially any information that a solver might request from a model.
%
\texttt{NLPModels.jl} \cite{orban-siqueira-nlpmodels-2020} is the core package that defines the abstract type \texttt{AbstractNLPModel} and introduces the API, including in-place and out-of-place evaluation of the objective gradient, sparse Jacobian and Hessian matrices as well as operators for matrix-free implementations.
%
The user can hand-code derivatives, use automatic differentiation \cite{orban-siqueira-nlpmodels-2021}, or use JSO interfaces to optimization modeling languages such as AMPL \cite{fourer2003ampl}, CUTEst \cite{cutest}, or JuMP \cite{jump}.
%
Hence, solvers can be designed to rely on the API's behavior independently of the problem's origin. 

JSO defines a minimal set of rules for a solver to be compliant:
(i) The input is an instance of \texttt{AbstractNLPModel};
(ii) the output is a \texttt{GenericExecutionStats}, as implemented in \texttt{SolverCore.jl} \cite{orban-siqueira-solvercore-2021}, that contains the solution, optimal value, elapsed time, number of iterations, primal and dual feasibility, etc.
%
There are a growing number of JSO-compliant solvers accessible from registered Julia packages for unconstrained and bound-constrained problems \cite{Dussault_AdaptiveRegularization_jl_A_unified,orban-siqueira-jsosolvers-2021}, equality-constrained problems \cite{migot2022dcisolver,Migot_FletcherPenaltySolver_jl_2022,Orban_A_regularization_method_2020}, and problems of the form \eqref{eq:nlp} in \cite{percival-jl,shin2023accelerating}.
Most of those solvers also have variants for nonlinear least squares.
A solver for convex quadratic programs is in \texttt{RipQP.jl} \cite{orban-leconte-ripqp-2020}.
The solvers mentioned are pure Julia implementations, but there also exist thin wrappers to well-established solvers such as \texttt{Artelys Knitro} \cite{byrd2006k} and \texttt{Ipopt} \cite{wachter2006implementation}.

The aim of \texttt{JSOSuite.jl} is to provide a unique interface to navigate among the available solvers and select the best one without knowing all the JSO-compliant solvers.
%
There is a strong emphasis on justifying the choice of solvers, as we believe it helps the overall engineering process from an application to modeling and solving a problem.
%
Connecting modeling tools and solvers into a unique entry point simplify greatly prototyping and choosing the right solver.

\section{Functionalities}

The main function exported by \texttt{JSOSuite.jl} is \texttt{minimize}, which, given an instance of \texttt{AbstractNLPModel}, selects an appropriate solver and locally minimizes the problem, e.g. Code~\ref{code:minimze}.

\begin{lstlisting}[
    language = Julia, 
    numbers=left,
    caption = {\texttt{minimize} takes any NLPModels compatible input.},
    label = {code:minimze}
]
using OptimizationProblems, JSOSuite
nlp = OptimizationProblems.PureJuMP.kirby2()
stats = minimize(nlp)

\end{lstlisting}

Internally, a \texttt{DataFrame}, \texttt{JSOSuite.optimizers}, contains all the useful information regarding each solver such as its name, package of origin, highest derivative used, type of problems handled, whether it accepts arbitrary arithmetic types, etc.
%
The function \texttt{JSOSuite.select\_optimizers} analyzes an instance of \texttt{AbstractNLPModel} and returns a \texttt{DataFrame} with the available solvers, along with the information that guided their selection.

The solvers are then selected by identifying the arithmetic type, type of constraints (unconstrained, bound-constrained, equality-constrained, inequalities, linear/nonlinear), objective function (linear, quadratic, linear/nonlinear least squares, nonlinear), and highest accessible derivative (1st order, 2nd order, or 2nd order matrix-free).
Finally, the user can either choose to call \texttt{minimize} with the appropriate solver or use the corresponding package directly.
%
The current strategy is to select the most specialized solver. Further heuristics are being researched to improve the selection process when multiple solvers are available for a given problem.

An important feature common to most JSO-compliant solvers is the possibility to run in-place solve, i.e., it is possible to pre-allocate the output and the storage used during the iterations to run the optimizer allocation free; see Code~\ref{code:solve!}.
This is of great interest as it shows memory efficiency and allows re-solving problems without storage overhead.
\begin{lstlisting}[
    language = Julia, 
    numbers=left,
    caption={\texttt{JSOSuite.solve!} can be used to re-solve without additional memory allocation.},
    label={code:solve!}
]
using NLPModelsTest, Percival, SolverCore
nlp = NLPModelsTest.HS6()
solver = PercivalSolver(nlp)
stats = GenericExecutionStats(nlp)
solve!(solver, nlp, stats)
SolverCore.reset!(solver)
@allocated solve!(solver, nlp, stats) # = 0
\end{lstlisting}

A list of parameters common to all JSO-compliant solvers is maintained, and can be passed to \texttt{minimize}.
%
Combined with \texttt{SolverBenchmark.jl} \cite{orban-siqueira-solverbenchmark-2020}, it can seamlessly compare algorithms and generate data and performance profiles on collection of test problems.
Additionally, the package implements strategies that are classical for continuous optimization solvers such as methods to find a feasible initial guess or run a multi-start strategy.

\section{Concluding Remarks}

Julia’s JIT compiler is ideal for efficient scientific computing and optimization software, making it a natural choice for developing new solvers.
%
The JuliaSmoothOptimizers organization provides a comprehensive set of tools for large-scale continuous optimization, designed for ease of use by practitioners, researchers, and developers.
%
Actively maintained and constantly evolving, these tools include links to cutting-edge external solvers and implementations of promising new ones. 
%
\texttt{JSOSuite.jl} builds on these features to offer a user-friendly interface that simplifies solver selection and problem-solving in research-level continuous optimization.

\section*{Acknowledgement}

Tangi Migot is supported by an NSERC Alliance grant 544900-19 in collaboration with Huawei-Canada and NSERC PGS-D,
and Dominique Orban is partially supported by an NSERC Discovery Grant.

% **************GENERATED FILE, DO NOT EDIT**************

\bibliographystyle{juliacon}
\bibliography{ref.bib}


\end{document}

% Inspired by the International Journal of Computer Applications template
